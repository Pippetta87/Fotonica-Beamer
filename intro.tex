\section{A cosa mi serve?}

\section{RegLez 19}
\begin{frame}[allowframebreaks]{Reg Lez 19}
\begin{itemize}
 \item 18/02/2019 - Introduzione al corso: programma, libri di testo, modalità d'esame, congresso di fotonica. Concetti di base di fisica delle eterostrutture a semiconduttore: tipi di eterostruttura, formalismo della funzione inviluppo, autostati e densità di stati. Pozzi quantici e sottobande; dispersione, elementi di matrice, transizioni interbanda e intersottobanda, regole di selezione.
\item 22/02/2019 - Guadagno in un semiconduttore: elementi di matrice, densità di stati congiunta, condizione di guadagno per i potenziali chimici. Dipendenza da frequenza ed iniezione, effetto della dimensionalità. Dipendenza dalla temperatura e dalla corrente. Meccanismi di iniezione. Resonant tunneling e guadagno in un laser a intersottobanda.
\item 25/02/2019 - Condizione di soglia per cavità Fabry-Perot. Cenni di ottica gaussiana, condizioni di stabilità dei risonatori e diffrazione, casi particolari. Guide d'onda planari: modi TE e TM. Tipi di guide d'onda laser. Fattore di confinamento.
\item 08/03/2019 - Frequenza dei modi laser: gain pulling. Laser multimodo: spectral e spatial hole burning. Laser a semiconduttore: geometrie, materiali, configurazioni. VCSEL e laser in cavità esterna. Laser a quantum wire: tecniche di crescita, proprietà, e prestazioni. Laser a quantum quantum dot: tecniche di crescita, proprietà, e prestazioni. Laser a cascata quantica: guide d'onda, materiali e prestazioni, disegno degli iniettori, laser a singolo modo e stabilizzazione, regioni attive a superreticolo.
\item 11/03/2019 - Laser a cascata quantica: laser multicolori, laser a grandi lunghezze d'onda, guide d'onda a plasmone di superficie, risonatori whispering gallery. Laser THz: regioni attive e guide d'onda, problematiche. Mode-locking: teoria e tecniche (mode-locking passivo, attivo, self-focusing), cenni ai pettini di frequenza.
\item 15/03/2019 - Laser a feedback distribuito: concetto di base, derivazione delle equazioni dei modi accoppiati, condizione di soglia, frequenza dei modi, selettività; caso di coefficiente di accoppiamento reale e immaginario, laser pi-shifted. Laser a emissione verticale: condizione di soglia nei VCSEL, laser DFB del second'ordine.
\item 18/03/2019 - Fotonica terahertz: Auston switches, time-domain spectroscopy, applicazioni di imaging e spettroscopia. Sorgenti THz: photomixers, diodi Gunn e varactors, OPO.
\item 22/03/2019 - Fotonica THz; rivelatori. Modulazione diretta dei laser: oscillazioni di rilassamento e frequenza massima, frequency chirp. Modulatori Stark: principio di funzionamento. Semiconduttori come amplificatori: saturazione, configurazione MOPA.
\item 25/03/2019 - Introduzione alla Fisica dei Materiali 2D; esfoliazione meccanica; ''zoologia'' delle varie famiglie di materiali 2D; eterostrutture di van der Waals; struttura a bande del grafene; elettroni a massa nulla ed equazione di Dirac-Weyl; paradosso di Klein; assorbimento ottico lineare.
\item 28/03/2019 - Trasporto di carica nel grafene; trasporto vicino al punto di Dirac; ''electron-hole puddles'' (EHPs); origine delle EHPs. Teoria dell'assorbimento ottico lineare: assorbimento inter-band universale e assorbimento intra-banda. Teoria di Drude. Bloccaggio di Pauli. Cenni di spettroscopia ARPES. Effetti a molti corpi nello spettro ARPES.
\item 29/03/2019 - Propagazione in un mezzo non lineare: equazioni caratteristiche, problema del phase matching, tecnica del quasi-phase matching. Amplificazione parametrica e oscillatori parametrici: guadagno, condizione di soglia, accordabilità.
\item 01/04/2019 - Modulatori elettro-ottici: cristalli birifrangenti, polarizzazione ordinaria e straordinaria, regime di linearità, limitazioni alla velocità (transit time), modulatori in linea di trasmissione e SAW. Controlo coerente dell'assorbimento: CPA, CPT, sistemi in strong-coupling, sistemi 2D, simmetria PT.
\item 05/04/2019 - Microcavit\'a: Hamiltoniana di Jaynes-Cummings, oscillazioni di Rabi di vuoto e splitting di Rabi di vuoto. Primi esperimenti, polaritoni eccitonici di cavit\'a, polaritoni intersottobanda. Regime di accoppiamento ultraforte. Modulazione del campo di vuoto.
\item 12/04/2019 - Plasmonica nei limiti ritardati e non ritardati: dalle equazioni di Maxwell alla teoria della risposta lineare. Filosofia dei due diversi approcci; cenni della struttura a bande di bilayer graphene e trilayer graphene; bande piatte in twisted bilayer graphene; altri materiali di Dirac: fenomenologia degli isolanti topologici. Aspetti di materials science: da HgTe a WTe2. Dall'effetto Hall quantistico intero ai quantum spin Hall insulators; grafene artificiale; introduzione alla teoria della risposta lineare, suo significato e utilit\'a; esempio di accoppiamento: potenziale scalare e densit\'a elettronica. 
\item 15/04/2019 - Teoria della risposta lineare, derivazione della formula di Kubo; rappresentazione in termini degli autostati esatti; propriet\'a generali delle funzioni di risposta causali; assorbimento.
\end{itemize}
\end{frame}